\chapter*[Interlude]{Interlude}

At this point in the document, it can perhaps benefit the reader to make a small summary of what has been discussed so far and how I continue to delineate my scientific contributions.

Multi-domain semantic similarity is useful to compare resources whose description spans several domains of knowledge, and biomedical informatics is rich in such resources: \eg epidemiological surges can be described using diseases, symptoms, pharmaceutical drugs, geographical locations, \etc. Comparison of these resources is important to enable searching capabilities on the multidisciplinary datasets, while allowing a certain kind of ``fuzziness'' on this search (resources need not fully satisfy a user query but can, instead, be similar to it).

Semantic similarity has been traditionally developed for single ontologies. As of June~2015, some published works deal with the multi-ontology problem, but all of them use multiple ontologies of the \emph{same domain} of knowledge, in an attempt to complement the knowledge in one ontology with the knowledge in another. Multi-\emph{domain} semantic similarity, in opposition, is important to compare multi-domain resources, annotated with concepts not only form distinct ontologies but from different fields of knowledge. No published literature deals with this problem, as far as I know.

As such, multi-domain semantic similarity measures need to be developed and validated. This is the research focus which this document reports. On the one hand, we can use single-ontology measures to compare concepts from one entity with ``compatible'' concepts from the second entity, thus obtaining a set of similarity values that can be mathematically aggregated into a single similarity value (aggregative approach); on the other hand, we can integrate all the relevant ontologies in a single knowledge-base and use existing measures directly on top of it (integrative approach).

This multi-faceted task will be described in the next three chapters. \chpref{chap:data} describes three multidisciplinary datasets collected to serve as test cases for multi-domain semantic similarity. The multi-domain measures (aggregative and integrative approaches) are presented in \chpref{chap:multidomain}, along with their associated results, stemming from its application over the three datasets. Finally, \chpref{chap:technical} will examine the technical details of semantic similarity, with particular focus on an open-source software suite that I developed to assist in calculating semantic similarity using OWL ontologies.

